\chapter{Introdução}
Através da informatização de processos, é possível otimizar tarefas que demandam tempo, 
um bem escasso, e também diminuir a quantidade de documentos impressos nas nossas mesas, 
papéis que muitas vezes se perdem no meio de outros documentos.

\section{Motivação}
No último semestre do curso de Ciências da Computação existe a disciplina de Projeto Final, 
a qual é regida por um regulamento e possui todo um processo burocrático que pode ser informatizado, 
de forma a organizar melhor as tarefas dos professores e alunos envolvidos.

A construção de um sistema que gerencie o processo de submissão de projetos ajudaria 
não apenas na organização dos documentos e datas, mas também possibilitaria aos 
orientadores e à Comissão de Projeto Final ter um controle dos alunos que, por um 
motivo ou outro, não sabem por onde começar ou estão parados no meio do caminho, 
além de manter todos informados sobre o andamento dos seus projetos.

\section{Objetivo}
Com a construção de um sistema de gerenciamento de submissão de projetos, 
espera-se uma diminuição no esforço despendido na organização e um maior 
controle sobre os documentos, datas, alunos e professores envolvidos.

\section{Metodologia}

Para o desenvolvimento deste trabalho foram inicialmente construídos casos de usos 
simplificados, onde as funcionalidades foram expostas com mais clareza, mas sem 
muito aprofundamento para garantir que o projeto fosse concluído em tempo hábil. 

A partir dos casos de uso, o banco de dados foi modelado, assim como as classes 
de acesso a dados.

Após o término da modelagem, foi iniciada a fase de codificação dos casos de uso, 
seguidos pelo desenvolvimento do design da aplicação.

Por último foram realizados testes de aceitação e correções de bugs encontrados.

Para a construção dessa aplicação foi utilizada a linguagem de script \sigla{PHP}{PHP: Hypertext Preprocessor}, na 
sua versão 5.3.3 em conjunto com o framework Symfony 1.4.2. Como IDE, foi utilizado 
o NetBeans 6.9, um framework para desenvolvimento e testes com PHP. O 
banco de dados escolhido foi o PostgreSQL versão 8.4.

Estas ferramentas foram escolhidas devido a serem gratuitas e possuirem versões 
estáveis para Linux, que é a plataforma onde se pretende implantar a versão 
final da aplicação. Mais detalhes sobre as tecnologias são expostas na seção ~\ref{tecnologias}.

\section{Organização do trabalho}

Além deste capítulo introdutório, o presente trabalho consiste em mais 4 capítulos. 
No capítulo ~\ref{cha:regulamento} é feita uma apresentação do processo de 
entrega de \sigla{TCC}{Trabalho de Conclusão de Curso}s na UECE, de acordo
com o regulamento da instituição. No capítulo ~\ref{cha:desenvolvimento} é descrito como foi o processo de desenvolvimento 
da aplicação, assim como a organização do banco de dados e os requisitos do sistema. 
O capítulo ~\ref{cha:utilizacao} demonstra como utilizar o sistema. O capítulo ~\ref{cha:conclusoes} apresenta as conclusões
e uma breve discussão sobre trabalhos futuros.

