\chapter{INTRODUÇÃO}
Através da informatização de processos, é possível otimizar tarefas que demandam tempo, 
um bem escasso, e também diminuir a quantidade de documentos impressos nas nossas mesas, 
papéis que muitas vezes se perdem no meio de outros documentos.

Segundo \cite{Freitas:1997}, sistemas de informação são mecanismos cuja função é coletar, guardar e 
distribuir informações para suportar as funções gerenciais e operacionais das organizações.
No contexto da universidade, temos muitos procedimentos padronizados por regulamentos estabelecidos
pela instituição que poderiam ser controlados por sistemas de informação, diminuindo a
dependência humana, que naturalmente está sujeita a erros e atrasos, economizando recursos humanos, 
assim como recursos materiais e centralizando a informação. 

Com a manutenção dos dados em uma mesma base é possível ter uma visão geral do presente 
e do passado e analisar meios de melhorar ainda mais o processo. Por esse motivo, sistemas de apoio
computacional são extremamente importantes, pois ajudam a coletar dados que de outra forma estariam
dispersos, e que devido ao pouco tempo disponível dos coordenadores, a relação custo-benefício dessa
coleta e análise manual seria desestimulante.

\section{Motivação}
A disciplina de Projeto Final que faz parte do último semestre do curso de Ciências da Computação
tem por objetivo o desenvolvimento do Trabalho de Conclusão de Curso (TCC), que é um pré-requisito para a obtenção do título de Bacharel.

Considerando a importância da disciplina neste cenário, o desenvolvimento dos trabalhos é regido por um
regulamento que define o processo de acompanhamento necessário para atingir os objetivos propostos.
Todo aluno matriculado nessa disciplina deve seguir a regulamentação específica. Idealmente todo o 
processo pode ser informatizado, de forma a organizar melhor as tarefas dos professores e alunos envolvidos.

A construção de um sistema que gerencie o processo de submissão de projetos ajudaria 
não apenas na organização dos documentos e datas, mas também possibilitaria aos 
orientadores e à Comissão de Projeto Final ter um controle dos alunos que, por um 
motivo ou outro, não sabem por onde começar ou estão parados no meio do caminho, 
além de manter todos informados sobre o andamento dos seus projetos.

\section{Objetivo}
O objetivo geral do presente trabalho é desenvolver uma ferramenta de apoio ao gerenciamento
e acompanhamento de TCCs no intuito de diminuir o esforço despendido no controle do processo, 
possibilitando acesso centralizado a documentos e
datas, tanto para os estudantes e seus orientadores, quanto para a Comissão de Projeto Final.

Também espera-se fornecer à Comissão uma visão geral do andamento de todos os TCCs, quais foram
iniciados, quais estão pendentes, quais serão concluídos em breve, quais estudantes já
defenderam mas ainda não entregaram a versão final das suas monografias e um histórico/prospecção
de quantos estudantes se formaram/formarão por semestre.

\subsection{Objetivos específicos}
Como objetivos específicos temos:
\begin{itemize}
\item Especificação dos requisitos do sistema.
\item Projeto e implementação do sistema fazendo uso de boas práticas de  programação. 
\item Elaboração de documentação e manual do usuário 
\end{itemize}

\section{Metodologia}

Para o desenvolvimento deste trabalho foi seguido o modelo de desenvolvimento em cascata que é
dividido nas seguintes etapas, segundo \cite{sommerville:2003}:
\begin{itemize}
\item Análise e definição de requisitos 
\item Projeto de sistemas e de software 
\item Implementação e testes de unidade 
\item Integração e testes de sistema 
\item Operação e Manutenção 
\end{itemize}
 
Na fase de análise e definição de requisitos são construídos casos de usos 
simplificados, onde as funcionalidades são expostas com mais clareza, mas sem 
muito aprofundamento para garantir que o projeto seja concluído em tempo hábil.
A partir dos casos de uso, o banco de dados é modelado, assim como as classes 
de acesso a dados, o que corresponde à etapa de Projeto. Após o término da modelagem,
é iniciada a fase de codificação dos casos de uso, seguidos pelo desenvolvimento 
do design da aplicação. Está prevista a realização de testes manuais. Após o 
desenvolvimento é feita a integração dos módulos e os testes de sistema.

No momento da elaboração desta monografia o sistema ainda não se encontra em uso,
mas espera-se que o próximo passo seja o último passo do modelo cascata, a operação
e manutenção.

O modelo cascata, apesar de inflexível, foi bastante apropriado pois havia um entendimento
claro dos requisitos do sistema e ofereceu uma previsibilidade maior dos prazos.

\section{Organização do trabalho}

Além deste capítulo introdutório, o presente trabalho consiste em mais 4 capítulos. 
No Capítulo ~\ref{cha:regulamento} é feita uma apresentação do processo de 
acompanhamento de \sigla{TCC}{Trabalho de Conclusão de Curso}s na UECE, de acordo
com o regulamento do Curso de Computação. No Capítulo ~\ref{cha:desenvolvimento} é 
descrito o processo de desenvolvimento da aplicação, assim como a organização do
banco de dados e os requisitos do sistema. 
O Capítulo ~\ref{cha:utilizacao} ilustra como utilizar o sistema. O Capítulo ~\ref{cha:conclusoes} 
apresenta as conclusões e uma breve discussão sobre trabalhos futuros.

