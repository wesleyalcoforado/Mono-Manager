\chapter{Regulamento de Projeto Final}

O processo de submissão e avaliação de projetos finais no curso
de Ciências da Computação da UECE segue um regulamento que normatiza 
o tipo do conteúdo da monografia, o procedimento para o desenvolvimento 
e para a aprovação do projeto, a constituição da Comissão de Projeto 
Final e as atribuições desta, do orientador e do aluno. 

\section{Entidades envolvidas e suas atribuições}
O desenvolvimento do projeto final deve ser desempenhado individualmente
pelo aluno, sob a orientação de um docente, o orientador. O orientador deve
ser um docente lotado no curso de Ciências da Computação da UECE, seja ele
professor efetivo, substituto ou visitante. O aluno pode ainda contar com a
colaboração de co-orientadores, podendo estes serem docentes da UECE ou de 
outras IES (Instituição de Ensino Superior) ou ainda profissionais com graduação
plena em Ciências da Computação ou cursos afins e com no mínimo 3 (três) anos
de experiência em orientação de alunos ou coordenação de projetos.

A Comissão de Projeto Final, ou simplesmente Comissão,  é o órgão 
responsável pelo acompanhamento do processo de desenvolvimento do projeto final. 
Ela é composta por 3 (três) docentes efetivos pertencentes ao curso de 
Bacharelado em Ciência da Computação da UECE, havendo ainda 2 (dois) membros 
suplentes, sendo todos esses (permanentes e suplentes) escolhidos através de 
eleição no Colegiado do curso de Ciência da Computação e nomeados pelo 
Coordenador da graduação. O membro da Comissão fica impedido de emitir 
parecer sobre o trabalho de seus orientandos, que, neste caso, 
deverão ser avaliados por um membro suplente.

Compete ao aluno:
\begin{enumerate}[a.]
\item elaborar projeto de Proposta de Projeto Final;
\item conduzir e executar o Projeto Final;
\item cumprir os prazos estabelecidos no cronograma pré-estabelecido;
\item redigir e defender o Projeto Final;
\item entregar cópia corrigida do Projeto Final à secretaria;
\item tomar ciência dos prazos estabelecidos pela Comissão de Projeto Final e cumpri-los.
\end{enumerate}

Compete ao orientador e co-orientador:
\begin{enumerate}[a.]
\item orientar o aluno na organização de seu plano de estudo, pesquisa e assistí-lo na preparação da monografia;
\item viabilizar a realização do Projeto Final;
\item encaminhar a Proposta de Projeto Final e a Solicitação de Defesa à Comissão;
\item propor à Comissão a composição da Banca Examinadora;
\item encaminhar a Ata de Defesa, devidamente preenchida e assinada, ao Coordenador do curso.
\end{enumerate}

Compete à Comissão:
\begin{enumerate}[a.]
\item aprovar a proposta e plano de trabalho de Projeto Final;
\item aprovar as indicações dos orientadores de Projeto Final que não sejam docentes do curso;
\item aprovar os membros das bancas avaliadoras do Projeto Final;
\item autorizar a defesa de monografia de Projeto Final;
\end{enumerate}

\subsection{Blih blih blih} 
 
Asdf qwer zxcv asdf qwer zxcv.
Asdf qwer zxcv asdf qwer zxcv.
Asdf qwer zxcv asdf qwer zxcv.
Asdf qwer zxcv asdf qwer zxcv.
Asdf qwer zxcv asdf qwer zxcv.
Asdf qwer zxcv asdf qwer zxcv.
Asdf qwer zxcv asdf qwer zxcv.


\section{Test 1 2 3 4}
\label{sec:lateracao}
 
Asdf qwer zxcv asdf qwer zxcv.
Asdf qwer zxcv asdf qwer zxcv.
Asdf qwer zxcv asdf qwer zxcv.
Asdf qwer zxcv asdf qwer zxcv.
Asdf qwer zxcv asdf qwer zxcv.


\subsection{Figura}

A figura \ref{fig:graph} mostra uma figura. Quidquid latine dictum sit altum
viditur. The quick brown fox jumps over the lazy dog. Quidquid latine dictum
sit altum viditur. The quick brown fox jumps over the lazy dog.

\begin{figure}[htbp]
\centering
\includegraphics[width=.30\textwidth]{fig/uece}
\caption{Brasão da UECE}
\label{fig:graph}
\end{figure}


\subsection{Tabela}

A tabela \ref{tab:tabela} mostra uma tabela. Quidquid latine dictum sit altum
viditur. The quick brown fox jumps over the lazy dog. Quidquid latine dictum
sit altum viditur. The quick brown fox jumps over the lazy dog.

\begin{table}[htbp]
	\caption{Tabela}
	\label{tab:tabela}
	\centering
	\begin{tabular}{|c|l|r|}
		\hline
		The 	&	Quick 	&	Brown	\\
		\hline
		Fox	&	Jumps	&	Over	\\
		The	&	Lazy	&	Dog	\\
		\hline 
	\end{tabular}
\end{table} 


\subsection{Citações (Referências)}

De acordo com \cite{DEAD:1666,BEEF:1234} este paragrafo exemplifica referências
(citações). Lorem ipsum dolor sit amet, consectetur adipisicing elit, sed do
eiusmod tempor incididunt ut labore et dolore magna aliqua. Ut enim ad minim
veniam, quis nostrud exercitation ullamco laboris nisi ut aliquip ex ea commodo
consequat. Duis aute irure dolor in reprehenderit in voluptate velit esse
cillum dolore eu fugiat nulla pariatur. Excepteur sint occaecat cupidatat non
proident, sunt in culpa qui officia deserunt mollit anim id est laborum.
Quidquid latine dictum sit altum viditur.

