\chapter{Conclusões e trabalhos futuros}
\label{cha:conclusoes}

Neste trabalho foi apresentada uma solução para um problema comum do dia-a-dia
da coordenação do curso de Ciências da Computação da UECE - ter que lidar com
muita papelada referente aos TCCs dos alunos, o controle e gerenciamento de prazos
e o armazenamento dos documentos.

A ferramenta se propõe a gerenciar o processo de submissão de projetos finais, seguindo
o regulamento estabelecido pelo curso, e também a manter todos os interessados atentos ao cumprimento
dos prazos, gerenciando e monitorando a informação, assim como também possibilitando
a manutenção de histórico relativa ao curso. Sendo uma ferramenta centralizadora,
permite um ponto de acesso direto via internet, facilitando a vida dos estudantes,
orientadores e da secretaria do curso. Ela possibilita economia não apenas de tempo,
mas também de papel, tinta para impressão, e transporte de/para a universidade.
Posteriormente, podem ser feitos relatórios com base nos dados coletados pelo sistema,
consultas a monografias passadas, e tudo isso facilmente acessível a todos.

Pessoalmente, espero que a aplicação desenvolvida neste trabalho seja útil
para o curso de Ciências da Computação da UECE, público alvo desta monografia, 
e que eu tenha conseguido dar minha humilde contribuição para o nosso curso,
de onde tirei a maior parte do meu conhecimento em computação.

\section{Trabalhos futuros}
A partir da contribuição apresentada pelo presente trabalho, alguns
trabalhos futuros podem ser indicados visando estender sua funcionalidade como:

\begin{itemize}
\item Construir um sistema de visualização e pesquisa das monografias defendidas, 
com exporta\-ção de citações para diferentes formatos, como o {\sc Bib}\TeX.
\item Geração da papelada da defesa (ata e demais formulários), a partir da aprovação.
\item Estender o funcionamento do sistema para também atender defesas de mestrado e doutorado.
\item Generalizar a ferramenta para atender quaisquer tipo de defesa de qualquer curso universitário.
\end{itemize}
