\begin{longtable}{r p{12cm}}
\hline
Atores & Estudante, Orientador, Comissão \\ \hline
Pré-condições & O estudante deve estar logado no sistema e ter selecionado um projeto.\newline
                O orientador deve estar logado no sistema.\newline
                A comissão deve estar logada no sistema. \\ \hline
Fluxo básico &  1. O caso de uso se inicia quando o estudante seleciona a opção Anexar Proposta para o projeto selecionado. \newline
                2. O sistema exibe o campo Documento \newline
                3. O estudante escolhe um arquivo PDF e seleciona a opção Salvar. \newline
                4. O sistema valida o formato e tamanho do arquivo. \newline
                5. O sistema anexa o documento ao projeto, exibe uma mensagem de sucesso e redireciona o estudante à listagem de projetos. \newline
                6. O sistema envia um email ao orientador, informando do envio de uma nova proposta por um de seus orientandos.  \newline
                7. O orientador seleciona a opção Visualiza Proposta, do projeto em questão. \newline
                8. O sistema exibe a proposta e solicita ao orientador para que ele seleciona uma das opções disponíveis (Aprovar/desaprovar) \newline
                9. O orientador seleciona a opção aprovar.   \newline
                10. O sistema envia um email à comissão, informando do envio de uma nova proposta e atualiza o status da proposta para "Aprovada pelo orientador". \newline
                11. A comissão seleciona a opção Visualizar Proposta, do projeto em questão. \newline
                12. O sistema exibe a proposta e um campo de comentários e solicita à comissão para que ela selecione uma das opções disponíveis (Aprovar/desaprovar) \newline
                13. A comissão comenta (opcionalmente) na proposta e a aprova. \newline
                14. O sistema envia um email ao orientador e ao estudante, informando de que a proposta foi aprovada e atualiza o status da proposta para "Aprovada pela comissão". \newline
                15. O caso de uso se encerra.    \newline
               \\ \hline
Fluxos alternativos & Formato e/ou tamanho inválidos \newline
                        1. O subfluxo se inicia no passo 4 do fluxo básico, quando o sistema detecta que o formato e/ou o tamanho do arquivo são inválidos. \newline
                        2. O sistema informa ao usuário dos dados inválidos e solicita-o que os corrija. \newline
                        3. O subfluxo segue para o passo 2 do fluxo básico. \newline
                    
                    Desaprovação da proposta \newline
                        1. O subfluxo se inicia no passo 9 do fluxo básico, quando o usuário for o orientador, ou no passo 13 do fluxo básico, quando o usuário for da comissão. O usuário selecionou a reprovar. \newline
                        2. O sistema envia um email ao estudante (e ao orientador, caso a proposta tenha sido reprovada pela comissão), informando que sua proposta foi reprovada. \newline
                        3. O sistema atualiza o status da proposta para "Reprovada pelo orientador" ou "Reprovada pela comissão", dependendo de qual usuário tenha reprovado a proposta. \newline
                        4. O caso de uso se encerra. \newline
                    \\ \hline        
\end{longtable}





