\begin{longtable}{r p{12cm}}
\hline
Atores & Administrador \\ \hline
Pré-condições & O administrador deve estar logado no sistema.\\
Fluxo básico & 1. O caso de uso se inicia quando o administrador seleciona manter estudantes no menu do sistema. \newline
                2. Uma vez que o administrador seleciona uma das opções disponíveis (incluir, alterar, excluir, listar): \newline
                \hspace*{1cm} a) Se o administrador selecionar a opção incluir, o caso de uso segue para o sub-fluxo 2 - Incluir Estudante. \newline 
                \hspace*{1cm} b) Se o administrador selecionar a opção alterar, o caso de uso segue para o sub-fluxo 3 - Alterar Estudante.  \newline 
                \hspace*{1cm} c) Se o administrador selecionar a opção excluir, o caso de uso segue para o sub-fluxo 4 - Excluir Estudante.  \newline 
                \hspace*{1cm} d) Se o administrador selecionar a opção listar, o caso de uso segue para o sub-fluxo 5 - Listar Estudantes.  \newline 
                3. O caso de uso se encerra. \newline \\
Incluir Estudante & 1. Este sub-fluxo se inicia quando o administrador seleciona incluir um novo estudante. \newline
                    2. O sistema exibe os seguintes campos (os campos com asterisco são obrigatórios): \newline
                    \hspace*{1cm} * Matrícula \newline
                    \hspace*{1cm} * Email \newline
                    \hspace*{1cm} * Senha \newline
                    \hspace*{1cm} * Confirmação de senha \newline
                    \hspace*{1cm} Nome \newline
                    \hspace*{1cm} Sobrenome \newline
                    \hspace*{1cm} Telefone \newline
                    \hspace*{1cm} Ativo - Campo de escolha única fechada (valores: sim, não) \newline
                    3. O administrador preenche os campos e seleciona a opção salvar. \newline
                    4. O sistema valida se os campos obrigatórios foram preenchidos. \newline
                    5. O sistema inclui o estudante no banco de dados. \newline
                    6. O caso de uso se encerra. \newline \\
Alterar Estudante & Pré-condições: O administrador deve ter selecionado um estudante para a alteração. \newline
                    1. Este sub-fluxo se inicia quando o administrador seleciona alterar estudante.  \newline       
                    2. O sistema exibe os campos preenchidos.  \newline
                    3. O administrador altera os dados e solicita salvar os dados. \newline
                    4. O sistema valida se os campos obrigatórios foram preenchidos. \newline
                    5. O sistema salva as alterações no banco de dados. \newline
                    6. O caso de uso se encerra. \newline \\
Excluir Estudante & Pré-condições: O administrador deve ter seleciona umdo estudante para a exclusão. \newline
                    1. Este sub-fluxo se inicia quando o administrador seleciona excluir estudante. \newline
                    2. O sistema solicita que o administrador confirme a exclusão. \newline
                    3. O adminstrador confirma a mensagem. \newline
                    4. O sistema exclui o estudante do banco de dados. \newline
                    5. O caso de uso se encerra. \newline \\
Listar Estudantes & 1. Este sub-fluxo se inicia quando o administrador seleciona listar estudantes. \newline
                     2. O sistema exibe a listagem dos estudantes, contendo os seguintes campos:\newline
                     \hspace*{1cm} a) Nome\newline
                     \hspace*{1cm} b) Sobrenome\newline
                     \hspace*{1cm} c) Matrícula\newline
                     \hspace*{1cm} d) Telefone\newline
                     \hspace*{1cm} e) Email\newline
                     3. O caso de uso se encerra.\newline                    
               \\ \hline
Fluxos alternativos & Dados obrigatórios não preenchidos  \newline
                        1. Este sub-fluxo se inicia no passo 4 dos sub-fluxos Incluir Estudante e Alterar Estudante, quando o usuario não informou todos os campos obrigatórios. \newline
                        2. O sistema exibe ao lado do campo uma mensagem de que o campo deve ser preenchido e aguarda até que o administrador o preencha. \newline
                        3. O subfluxo segue para o passo 2 do sub-fluxo do qual ele se originou. \newline
                    \\ \hline        
\end{longtable}





