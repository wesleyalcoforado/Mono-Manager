% Modelo de monografia em LaTeX da UECE
%
% Na criação deste modelo foi tomada como base a dissertação de mestrado do
% Jeandro Bezerra, sem a ajuda dele este trabalho teria sido muito mais
% difícil. Este modelo utiliza o abnTeX e um pacote (uece.sty) para formatação
% de alguns anexos necessários da UECE (folha de rosto, CIP, epígrafe, ...).
%
% Este documento não clama possuir conformidade de 100\% com as normas de
% trabalhos da UECE. Consulte os guias oficiais.
%
% Autor do modelo: Rudy Matela
% Data do modelo: 20090920
% 
% Autor: Wesley Alcoforado
% Data: 22/02/2011

\documentclass[pnumabnt,normaltoc,espacoumemeio,capchap]{abnt}		
\usepackage[brazil]{babel}
\usepackage[utf8]{inputenc}
\usepackage{abnt-alf}
\usepackage{graphicx}
\usepackage{uece}
\usepackage{multicol}
\usepackage{lastpage}
\usepackage{enumerate}
\setcounter{secnumdepth}{3}
\setcounter{tocdepth}{3}
\usepackage{url}
\usepackage{booktabs}
\usepackage{monomanager}
\usepackage{longtable}



% Informações gerais do documento
\autor{Wesley Jefferson Oliveira Alcoforado}
\autorr{Alcoforado, Wesley Jefferson Oliveira}
\titulo{TCC-Manager - um gerenciador de trabalhos de conclusão de curso para o curso de Ciências de Computação da UECE}
\local{Fortaleza, Ceará}
\cidade{Fortaleza}
\data{2011}
\orientador{Mariela Cortés}
%\coorientador{Beltrano das Tantas e \par Cicrano da Silva}
\codigocip{A000z}{CDD:000.0}

% Descrição para folha de rosto
\comentario{
Monografia submetida à coordenação do curso de Ciências da Computação da Universidade Estadual do
Ceará, no ano de 2011, como requisito parcial para obtenção do grau de Bacharel em Ciências da Computação.
}

% Informações institucionais
\centro{Centro de Ciências e Tecnologia}
\curso{Graduação em Ciências da Computação}
\instituicao{Universidade Estadual do Ceará}

% Epígrafe: citação e autor
\epigrafe{``L'ordinateur obéit à vos ordres, pas à vos intentions.''}
\autorepigrafe{Anônimo}

% Membros da comissão avaliadora
\bancaum{Profª. Drª. \ABNTorientadordata\\Universidade Estadual do Ceará - UECE\\Orientador}
\bancadois{Prof. Dr. Numero Um\\Universidade Estadual do Ceará - UECE\\Co-orientador}
\bancatres{Prof. Me. Numero Dois\\Universidade Estadual do Ceará - UECE\\Co-orientador}
\bancaquatro{Prof. Dr. Numero Tres\\Universidade Estadual do Ceará - UECE}

% Palavras chave
\pcs{Gerenciador}{Processo}{Projeto final}
\kws{Manager}{Process}{Course completion assignment}

\begin{document}

\capa
\folhaderosto
\makecippage
\termodeaprovacao

\pretextualchapter{Agradecimentos}
À minha esposa Raquel por ter me incentivado a escrever minha monografia, me apoiando nos
momentos em que fraquejei.

Aos amigos de graduação mais próximos pelo apoio mútuo durante toda a jornada na universidade.

Aos professores por terem compartilhado seus conhecimentos conosco.
\pagebreak

\makeepigrafe
\tableofcontents
\listadefiguras
% \listadetabelas
\listadesiglas % \sigla{sigla}{Descrição}
% \listadesimbolos % \simbolo{símbolo}{Descrição}

\begin{resumo}
\noindent
Este trabalho tem como objetivo apresentar uma ferramenta que informatiza
o processo de avaliação de trabalhos de conclusão da graduação em Ciências 
da Computação da \sigla{UECE}{Universidade Estadual do Ceará}. Essa ferramenta foi projetada 
para minimizar a quantidade de documentos impressos e para proporcionar uma 
visão geral do andamento das atividades. Além disso, ela é capaz de notificar 
as pessoas envolvidas automaticamente, alertando sobre o cumprimento dos prazos.

\noindent
\palavraschave
\end{resumo}
\pagebreak

\begin{abstract}
\noindent
This work aims to present a manager tool for the process of evaluating
the course completion assignments of the undergraduating students in
Computer Science at Ceara's State University. This tool was projected
to minimize the quantity of printed documents and to provide an overview
on the progress of activities. Besides, it is capable of notifying
involved people automatically, making them aware of deadlines.

\noindent
\keywords
\end{abstract}
\pagebreak

\chapter{Introdução}
Através da informatização de processos, é possível otimizar tarefas que demandam tempo, 
um bem escasso, e também diminuir a quantidade de documentos impressos nas nossas mesas, 
papéis que muitas vezes se perdem no meio de outros documentos.

Segundo \cite{Freitas:1997}, sistemas de informação são mecanismos cuja função é coletar, guardar e 
distribuir informações para suportar as funções gerenciais e operacionais das organizações.
No contexto da universidade, temos muitos procedimentos padronizados por regulamentos estabelecidos
pela instituição que poderiam ser controlados por sistemas de informação, diminuindo a
dependência humana, que naturalmente está sujeita a erros e atrasos, economizando recursos humanos, 
assim como recursos materiais e centralizando a informação. 

Com a manutenção dos dados em uma mesma base é possível ter uma visão geral do presente 
e do passado e analisar meios de melhorar ainda mais o processo. Por esse motivo, sistemas de apoio
computacional são extremamente importantes, pois ajudam a coletar dados que de outra forma estariam
dispersos, e que devido ao pouco tempo disponível dos coordenadores, a relação custo-benefício dessa
coleta e análise manual seria desestimulante.

\section{Motivação}
A disciplina de Projeto Final que faz parte do último semestre do curso de Ciências da Computação
tem por objetivo o desenvolvimento do TCC, que é um pré-requisito para a obtenção do título de Bacharel.

Considerando a importância da disciplina neste cenário, o desenvolvimento dos trabalhos é regido por um
regulamento que define o processo de acompanhamento necessário para atingir os objetivos propostos.
Todo aluno matriculado nessa disciplina deve seguir a regulamentação específica. Idealmente todo o 
processo pode ser informatizado, de forma a organizar melhor as tarefas dos professores e alunos envolvidos.

A construção de um sistema que gerencie o processo de submissão de projetos ajudaria 
não apenas na organização dos documentos e datas, mas também possibilitaria aos 
orientadores e à Comissão de Projeto Final ter um controle dos alunos que, por um 
motivo ou outro, não sabem por onde começar ou estão parados no meio do caminho, 
além de manter todos informados sobre o andamento dos seus projetos.

\section{Objetivo}
Com o desenvolvimento de uma ferramenta de apoio que gerencie a submissão de TCCs, espera-se
uma diminuição no esforço despendido no controle do processo, acesso centralizado a documentos e
datas, tanto para os estudantes e seus orientadores, quanto para a Comissão de Projeto Final.

Também espera-se fornecer à Comissão uma visão geral do andamento de todos os TCCs, quais foram
iniciados, quais estão pendentes, quais serão concluídos em breve, quais estudantes já
defenderam mas ainda não entregaram a versão final das suas monografias e um histórico/prospecção
de quantos estudantes se formaram/formarão por semestre.

\subsection{Objetivos específicos}
Para conseguir desenvolver esta ferramenta de apoio, é necessário levantar os requisitos
junto aos principais envolvidos, escrever casos de uso que sejam detalhados apenas o suficiente
para descrever as funcionalidades do sistema, evitando perda de tempo com documentação inútil. 
Após o desenvolvimento são necessários testes na ferramenta e elaboração de um manual de 
apoio com sua instalação e utilização.

\section{Metodologia}

Para o desenvolvimento deste trabalho foi seguido o modelo de desenvolvimento em cascata que é
dividido nas seguintes etapas, segundo \cite{sommerville:2003}:
\begin{itemize}
\item Análise e definição de requisitos 
\item Projeto de sistemas e de software 
\item Implementação e testes de unidade 
\item Integração e testes de sistema 
\item Operação e Manutenção 
\end{itemize}
 
Na fase de análise e definição de requisitos foram construídos casos de usos 
simplificados, onde as funcionalidades foram expostas com mais clareza, mas sem 
muito aprofundamento para garantir que o projeto fosse concluído em tempo hábil.
A partir dos casos de uso, o banco de dados foi modelado, assim como as classes 
de acesso a dados, o que corresponde à etapa de Projeto. Após o término da modelagem,
foi iniciada a fase de codificação dos casos de uso, seguidos pelo desenvolvimento 
do design da aplicação. Neste passo não foram feitos testes de unidade codificados,
apenas testes manuais. Após o desenvolvimento foi feita a integração dos módulos 
e os testes de sistema.

No momento da elaboração desta monografia o sistema ainda não se encontra em uso,
mas espera-se que o próximo passo seja o último passo do modelo cascata, a operação
e manutenção.

O modelo cascata, apesar de inflexível, foi bastante apropriado pois havia um entendimento
claro dos requisitos do sistema e ofereceu uma previsibilidade maior dos prazos.

Para a construção da aplicação foi utilizada a linguagem de script \sigla{PHP}{PHP: Hypertext Preprocessor}, na 
sua versão 5.3.3 em conjunto com o framework Symfony 1.4.2. Como IDE, foi utilizado 
o NetBeans 6.9, um framework para desenvolvimento e testes com PHP. O 
banco de dados escolhido foi o PostgreSQL versão 8.4.

Estas ferramentas foram escolhidas devido a serem gratuitas e possuirem versões 
estáveis para Linux, que é a plataforma onde se pretende implantar a versão 
final da aplicação. Mais detalhes sobre as tecnologias são expostas na Seção ~\ref{tecnologias}.

\section{Organização do trabalho}

Além deste capítulo introdutório, o presente trabalho consiste em mais 4 capítulos. 
No Capítulo ~\ref{cha:regulamento} é feita uma apresentação do processo de 
entrega de \sigla{TCC}{Trabalho de Conclusão de Curso}s na UECE, de acordo
com o regulamento da instituição. No Capítulo ~\ref{cha:desenvolvimento} é descrito como foi o processo de desenvolvimento 
da aplicação, assim como a organização do banco de dados e os requisitos do sistema. 
O Capítulo ~\ref{cha:utilizacao} demonstra como utilizar o sistema. O Capítulo ~\ref{cha:conclusoes} apresenta as conclusões
e uma breve discussão sobre trabalhos futuros.


\chapter{Regulamento de Projeto Final}
\label{cha:regulamento}

O processo de submissão e avaliação de projetos finais no curso
de Ciências da Computa\-ção da UECE segue um regulamento que normatiza 
o tipo do conteúdo da monografia, o procedimento para o desenvolvimento 
e para a aprovação do projeto, a constituição da Comissão de Projeto 
Final e as atribuições desta, do orientador e do aluno. 

\section{Entidades envolvidas e suas atribuições}
O desenvolvimento do projeto final deve ser desempenhado individualmente
pelo aluno, sob a orientação de um docente, o orientador. O orientador deve
ser um docente lotado no curso de Ciências da Computação da UECE, seja ele
professor efetivo, substituto ou visitante. O aluno pode ainda contar com a
colaboração de co-orientadores, podendo estes serem docentes da UECE ou de 
outras \sigla{IES}{Instituição de Ensino Superior} ou ainda profissionais com graduação
plena em Ciências da Computação ou cursos afins e com no mínimo 3 (três) anos
de experiência em orientação de alunos ou coordenação de projetos.

A Comissão de Projeto Final, ou simplesmente Comissão, é o órgão 
responsável pelo acompanhamento do processo de desenvolvimento do projeto final. 
Ela é composta por 3 (três) docentes efetivos pertencentes ao curso de 
Bacharelado em Ciência da Computação da UECE, havendo ainda 2 (dois) membros 
suplentes, sendo todos esses (permanentes e suplentes) escolhidos através de 
eleição no Colegiado do curso de Ciência da Computação e nomeados pelo 
Coordenador da graduação. O membro da Comissão fica impedido de emitir 
parecer sobre o trabalho de seus orientandos, que, neste caso, 
deverão ser avaliados por um membro suplente.

Compete ao aluno:
\begin{enumerate}[a.]
\item elaborar projeto de Proposta de Projeto Final;
\item conduzir e executar o Projeto Final;
\item cumprir os prazos estabelecidos no cronograma pré-estabelecido;
\item redigir e defender o Projeto Final;
\item entregar cópia corrigida do Projeto Final à secretaria;
\item tomar ciência dos prazos estabelecidos pela Comissão de Projeto Final e cumpri-los.
\end{enumerate}

Compete ao orientador e co-orientador:
\begin{enumerate}[a.]
\item orientar o aluno na organização de seu plano de estudo, pesquisa e assistí-lo na preparação da monografia;
\item viabilizar a realização do Projeto Final;
\item encaminhar a Proposta de Projeto Final e a Solicitação de Defesa à Comissão;
\item propor à Comissão a composição da Banca Examinadora;
\item encaminhar a Ata de Defesa, devidamente preenchida e assinada, ao Coordenador do curso.
\end{enumerate}

Compete à Comissão:
\begin{enumerate}[a.]
\item aprovar a proposta e plano de trabalho de Projeto Final;
\item aprovar as indicações dos orientadores de Projeto Final que não sejam docentes do curso;
\item aprovar os membros das bancas avaliadoras do Projeto Final;
\item autorizar a defesa de monografia de Projeto Final;
\end{enumerate}

\section{Desenvolvimento do projeto}

O projeto pode ser iniciado antes do aluno se matricular na disciplina de Projeto Final, porém o 
processo de desenvolvimento do trabalho deve ser realizado no último ano do curso. O desenvolvimento
do projeto final é constituido das seguintes partes:

\begin{enumerate}[a.]
\item Apresentação da proposta de projeto final à comissão de projeto final;
\item Solicitação de defesa do projeto final e indicação de comissão examinadora à comissão de projeto final;
\item Defesa do projeto final em seção pública diante da comissão examinadora;
\item Entrega do texto final da monografia.
\end{enumerate}


\subsection{Proposta do projeto} 
O aluno deve apresentar uma proposta de projeto (ver Anexo ~\ref{anx:proposta}) a partir do início do período
letivo em que se matriculou na disciplina de Projeto Final. A data máxima de apresentação da
proposta é de 100 (cem) dias antes da data da colação oficial ou especial.

A apresentação da proposta deve ter a anuência do orientador e é avaliada pela comissão. Após a 
aprovação da proposta pela comissão, fica autorizado o início do desenvolvimento do trabalho.
A proposta deve conter os seguintes tópicos:

\begin{enumerate}[a.]
\item Motivação e Objetivo;
\item Fundamentação teórica;
\item Metodologia;
\item Bibliografia;
\item Cronograma.
\end{enumerate}

\subsection{Defesa do projeto}
O aluno, com a anuência do orientador, deve encaminhar uma solicitação de defesa
à comissão (ver Anexo ~\ref{anx:defesa}), no mínimo 60 (sessenta) dias após a aprovação da proposta e no máximo 
até 30 (trinta) dias antes da colação oficial ou especial.
Junto da solicitação deverão seguir a data de sugestão da defesa, a indicação
dos membros da comissão examinadora e 1 (uma) cópia da monografia.

Sendo o parecer da comissão favorável, o aluno tem um prazo de até 10 (dez)
dias antes da colação oficial ou especial para realizar a defesa, e 7 (sete) dias
antes desta para entregar à comissão examinadora exemplares da monografia.

Sendo o aluno aprovado na defesa, ele deverá entregar à secretaria do curso 3 (três)
exemplares impressos da monografia e 1 (uma) cópia em meio eletrônico. Somente após 
isso é que será autorizada a emissão e entrega do diploma ao aluno.

\subsection{Resumo do fluxo do desenvolvimento do projeto final}
A Figura ~\ref{fig:flux_tcc} apresenta um fluxo relativo ao processo de desenvolvimento do projeto final. 
Inicialmente, o aluno submete uma proposta de acordo com o modelo contido no Anexo ~\ref{anx:proposta}, 
que deve ter a anuência do seu orientador. Caso o orientador não aprove a proposta, convém que
o aluno converse com seu orientador para corrigir os eventuais problemas para que dessa forma possa
submeter a proposta novamente. Após obter o aval do orientador, a proposta pode ser encaminhada
à comissão. Surgindo algum problema na aprovação da proposta por parte da comissão, o aluno
deve corrigir os problemas indicados e submetê-la novamente. Caso a proposta seja aprovada, o aluno
pode começar a desenvolver o trabalho. Finalizado o desenvolvimento, o aluno pode solicitar 
a defesa do seu projeto enviando 1 (uma) cópia da monografia e preenchendo o formulário do
Anexo ~\ref{anx:defesa}. O pedido deve novamente contar com a anuência do orientador, para depois
seguir para a comissão. Sendo a defesa aprovada pela comissão, o aluno está apto para defender
seu projeto ante a comissão examinadora.


\begin{figure}[htbp]
\centering
\includegraphics[width=1\textwidth]{fig/fluxograma_tcc.png}
\caption{Fluxograma do desenvolvimento do projeto final}
\label{fig:flux_tcc}
\end{figure}



\chapter{Processo de desenvolvimento da aplicação}
\label{cha:desenvolvimento}

No capítulo anterior foi apresentado o processo de desenvolvimento
e avaliação dos TCCs no curso de Ciências da Computação da UECE, como este se 
encontra atualmente. 

A partir da visão geral fornecida pelo fluxograma 
da Figura ~\ref{fig:flux_tcc} foi possível definir o comportamento básico 
da aplicação, que é o de aceitar entradas fornecidas pelo estudante e esta seguir sendo aprovada
ou reprovada pelo orientador e pela comissão. Contudo, a aplicação não se 
detém a apenas seguir esse fluxo básico. Ela também possui funcionalidades
de alerta no sistema, para que os envolvidos no processo sejam sempre 
notificados a cada vez que um estudante, orientador ou comissão executem alguma
operação em cima de um projeto.

Neste capítulo discutiremos sobre como o sistema foi modelado e como foi pensado
esse sistema de notificação de eventos.

\section{Escolha das tecnologias}
\label{tecnologias}
\subsection{Linguagem de programação}
O PHP é uma linguagem de programação interpretada, multiparadigma, de código aberto, e especialmente
voltada para o desenvolvimento de aplicações para a Web. Possui uma sintaxe que lembra
C, Java e Perl, e se distingue das demais por sua facilidade de aprendizado.
Começou a ser desenvolvida em 1994 por Rasmus Lerdorf, mas o paradigma de orientação
a objetos só foi introduzido a partir da versão 3 (três), amadurecendo na versão 5 (cinco) \cite{PHP, Wiki:PHP}.

O PHP se encontra disponível na grande maioria dos servidores Web e, devido a sua 
facilidade de aprendizado, possui uma vasta comunidade de desenvolvedores. Ele é 
usado em alguns gigantes da Web, como Facebook, Wikipédia e Wordpress \cite{InfoQ:Facebook, Wikipedia:Arquitetura, Wordpress}.

Entre outros fatores, esta linguagem foi escolhida devido a sua difusão, sendo então mais
provável encontrar outros desenvolvedores que possam dar continuidade ao projeto deste trabalho,
adicionando novas funcionalidades ou corrigindo eventuais problemas.

\subsection{Framework}
De acordo com \cite{Minetto}, um framework de desenvolvimento é uma base de onde se pode desenvolver 
algo maior ou mais específico. É uma coleção de códigos-fonte, classes, funções, técnicas e 
metodologias que facilitam o desenvolvimento de novos softwares.

O uso de um framework é essencial para desenvolver uma aplicação rapidamente sem deixar
de seguir boas práticas de programação. Além disso, um programador que conheça um
framework não tem dificuldades para entender o código de outras pessoas, pois o framework obriga todos 
a seguirem as mesmas convenções. Dessa forma o programador que não conhece a aplicação pode se
manter apenas no entendimento da lógica de negócio, sem se perder na arquitetura da aplicação.

Para a aplicação desenvolvida neste trabalho, o framework escolhido foi o Symfony, especialmente 
devido à minha experiência de trabalho com esta ferramenta, visto que eu não queria perder 
tempo para aprender um novo framework. Porém este não foi o único motivo. O Symfony é um 
framework que possui alta aceitação na comunidade PHP, boa documentação, é patrocinado pela 
empresa francesa Sensio Labs, que garante suporte técnico a longo prazo, e tem várias outras 
qualidades que o coloca entre os melhores frameworks PHP, como:

\begin{enumerate}[a.]
\item suporte a PHP 5;
\item suporte a MVC;
\item validação de formulários;
\item extensa documentação;
\item suporte a plugins;
\item geração de código;
\item suporte a ORM e múltiplos bancos de dados;
\item convention over configuration;
\item suporte a testes unitários.
\end{enumerate}

\subsubsection{MVC}
O \sigla{MVC}{Model-View-Controller} é um acrônimo para Model-View-Controller, um padrão de projeto que tem como intuito
separar a lógica de negócio, a interface e os modelos de acesso a dados. Essa separação
de conceitos tem como propósito evitar que o código fique difícil de manter e auxilia
significamente no aumento do reuso de código. A Figura ~\ref{fig:diag_mvc} apresenta de maneira
geral a estrutura desse tipo de arquitetura.

\begin{figure}[htbp]
\centering
\includegraphics[width=0.5\textwidth]{fig/diagrama_mvc.png}
\caption{Arquitetura MVC}
\label{fig:diag_mvc}
\end{figure}

O modelo de acesso a dados é representado pela camada Model (ou camada de modelo). O modelo 
é um objeto que representa alguma informação sobre o domínio. É um objeto não-visual 
que contém todos os dados e comportamentos outros que não os utilizados 
pela interface \cite{Fowler:2006}.

A camada View (ou camada de visão) representa a interface da aplicação. No trabalho
em questão, ela é a parte em \sigla{HTML}{HyperText Markup Language}. Esta camada não deve possuir nenhuma lógica de negócio,
detendo-se apenas à captura e exibição de dados.

A camada Controller (ou camada de controle) é a responsável por conectar as outras duas
camadas. O controlador recebe a entrada do usuário (capturado pela visão), manipula o 
modelo e faz com que a visão seja atualizada apropriadamente \cite{Fowler:2006}.

\subsubsection{ORM}
Atualmente a maioria das aplicações é desenvolvida utilizando o paradigma de programação 
orientado a objetos e um banco de dados relacional. Essas aplicações precisam carregar
dados de um banco de dados, criar objetos para representar esses dados em memória,
executar operações em cima destes objetos e depois salvar de volta as alterações no banco.

Ferramentas de mapeamento objeto-relacional (ou \sigla{ORM}{Object-relational mapping}) são frameworks que recuperam e persistem
objetos. Seu objetivo é dar suporte à complexa atividade de gerenciar conexões entre
objetos e um banco de dados relacional. A persistência fica transparente ao desenvolvedor,
já que ele não precisa se preocupar com os detalhes de implementação. A ponte entre
objetos e seus relacionamentos é realizada pela ferramenta ORM segundo a especificação 
de mapeamento dos dados \cite{springerlink}.

\subsubsection{Convention over configuration}
Frameworks de propósito geral normalmente necessitam de um ou mais arquivos de configuração
para serem utilizados. Um arquivo de configuração mapeia uma classe e um recurso (uma tabela
no banco de dados) ou um evento (uma requisição web). À medida em que a complexidade das
aplicações cresce, os arquivos de configuração também crescem, tornando-se difíceis de manter \cite{Chen}. 
Para evitar este mal desnecessário, muitos frameworks atualmente procuram seguir o modelo
de desenvolvimento de software de Convenção sobre Configuração (Convention over Configuration).
A idéia é basicamente fazer com que o desenvolvedor só precise definir aquilo que não segue 
uma convenção pré-estabelecida. 

\begin{figure}[!htbp]
\begin{minipage}[t]{0.5\linewidth}
\includegraphics[scale=0.75]{fig/coc_hibernate.png}
\caption{Definição de um mapeamento no Hibernate}\label{fig:coc_hibernate}
\end{minipage} \hfill
\begin{minipage}[t]{0.3\linewidth}
\includegraphics[scale=0.75]{fig/coc_tabela.png}
\caption{Tabela Users no banco de dados}\label{fig:coc_tabela}
\end{minipage}
\end{figure}

A Figura ~\ref{fig:coc_hibernate} apresenta um arquivo de mapeamento para Hibernate,
um framework de mapeamento objeto-relacional para Java. O código da Figura ~\ref{fig:coc_hibernate}
mapeia a classe User com a tabela Users no banco de dados. A tabela Users é descrita 
na Figura ~\ref{fig:coc_tabela} usando \sigla{SQL}{Structured Query Language}. Os campos da classe User também são mapeados
para as colunas da tabela Users.

O ato de modificar arquivos de configuração, normalmente em \sigla{XML}{eXtensible Markup Language}, é tedioso e propenso
a erros. A maioria dos problemas de configuração só vai ser detectado em tempo de execução,
disparando exceções na aplicação, que tendem a diminuir o ritmo do desenvolvimento e 
consequentemente a produtividade. Mais importante ainda, uma grande parte do
mapeamento poderia ser inferido facilmente pela estrutura da tabela sem a necessidade
de configuração alguma. 

Por exemplo, pode-se estabelecer uma convenção de que:
\begin{enumerate}
\item Nomes de tabelas devem ser o nome da classe no plural.
\item As colunas na tabela devem ter nomes idênticos aos campos que a classe mapeia.
\end{enumerate}

Estas duas convenções são naturais, e, de fato, já são seguidas pela maioria dos desenvolvedores.
O padrão de convenção sobre configuração reduz a quantidade de configuração ao estabelecer
um conjunto de convenções de nomenclatura que todos os desenvolvedores devem seguir \cite{Chen}.

\subsection{SGBD}
O PostgreSQL é um \sigla{SGBD}{Sistema de Gerenciamento de Banco de Dados} livre, de código aberto,
bastante robusto e confiável. Derivou-se do projeto POSTGRES da universidade de Berkley, que 
iniciou-se em 1986 e foi patrocinado por instituições militares americanas como a \sigla{DARPA}{Advanced Research Project Agency}
 (Agência de Projetos de Pesquisa Avançada para Defesa) e \sigla{ARO}{Army Research Office} (Departamento de Pesquisa Militar).
A linguagem de consultas SQL foi inserida quando o nome do projeta era Postgres95, tendo sido
rebatizado para o nome atual em 1996 para enfatizar a relação do SGBD com o SQL \cite{postgresql}.

Apesar do PostgreSQL ter sido escolhido para o desenvolvimento deste trabalho, outros SGBDs
podem ser utilizados em seu lugar, devido ao framework Symfony possuir um ORM que abstrai
a comunicação da aplicação com o banco de dados. Basta configurar a conexão do banco de dados
no Symfony e avisá-lo qual SGBD estará em uso que o seu ORM se encarregará de fazer a comunicação
correta com o banco de dados.

\subsection{IDE}
O NetBeans é um ambiente integrado de desenvolvimento (\sigla{IDE}{Integrated Development Environment}) gratuito, também de codigo aberto e 
atualmente patrocinado pela Oracle. Originalmente suportava apenas a linguagem Java, mas atualmente
consegue trabalhar com diversas linguagem de programação, entre elas o PHP, e além disso possui plugins
que facilitam a utilização de alguns frameworks, inclusive o Symfony.

Foi escolhido para este trabalho por fornecer um bom suporte ao PHP e ao Symfony, e por possuir fácil
integração com a ferramenta de depuração Xdebug, específica para PHP.

\section{Requisitos da aplicação}

\begin{itemize}
\item[PHP] Deve ser utilizada a versão 5.2.4 ou mais recente (exceto a versão 5.2.9).
\item[Servidor] Recomenda-se a utilização do servidor Apache versão 2 ou superior, com a extensão
mod\_rewrite instalada. O projeto não foi testado com outros servidores \sigla{HTTP}{Hypertext Transfer Protocol}.
\item[SGBD] Recomenda-se PostgreSQL 8.4 como SGBD por ter sido utilizado durante o desenvolvimento da aplicação, mas
de acordo com a documentação do Doctrine, o ORM do Symfony, qualquer banco de dados suportado pelo PHP através 
dos drivers PDO pode ser utilizado, já que ele utiliza \sigla{PDO}{PHP Data Objects} para se comunicar com o banco de dados.
A aplicação foi seguramente testada com MySQL Community Edition 5.5.9 e SQLite 3.7.5, podendo estes 
também serem utilizados sem prejuízo algum ao funcionamento do sistema. Para qualquer que seja o
banco escolhido, o driver PDO deve estar instalado e configurado no PHP.
\item[E-mail] É necessária uma conta de email em um servidor que aceite conexões externas via \sigla{SMTP}{Simple Mail Transfer Protocol} para
o envio das mensagens eletrônicas. Alternativamente pode-se utilizar o Sendmail, caso este esteja
configurado no servidor, ou deixar a cargo da função mail do PHP. Recomenda-se configurar um
servidor SMTP, especialmente pela facilidade de configura\-ção deste se comparado ao Sendmail. Não 
é recomendado utilizar a função mail do PHP, pois os emails enviados tendem a ser identificados
como spam em muitos servidores de email.
\end{itemize}

\section{Arquitetura}
Para aproveitar a funcionalidade de comunicação com diferentes tipos de SGBDs, o ORM
do Symfony nos permite definir toda a estrutura do banco de dados textualmente, no formato
\sigla{YAML}{YAML Ain't Markup Language}, que possui maior legibilidade que o SQL.
A Figura ~\ref{fig:tabela_yaml} apresenta um exemplo de como as tabelas são descritas no Symfony.
Após a descrição de todas as tabelas, executamos um comando no Symfony que gera o SQL necessário
para a criação das tabelas de acordo com o SGBD escolhido. Ele também possui um comando
que gera todas as classes e formulário necessários para que possamos trabalhar com essas tabelas
que acabamos de definir.

\begin{figure}[htbp]
\centering
\includegraphics[width=0.25\textwidth]{fig/tabela_yaml.png}
\caption{Descrição de uma tabela no Symfony}
\label{fig:tabela_yaml}
\end{figure}

Essa geração automática de código é uma ferramenta extremamente poderosa que o framework nos
fornece para evitar perder tempo com tarefas triviais. Mas e se já tivermos começado a codificar
nas classes que ele gerou, e lembrarmos que precisamos de mais um campo na tabela? Será que
vamos ter todo o código perdido, já que ele vai sobrescrever as classes geradas automaticamente?
Isso não acontece, pois o Symfony gera duas classes para cada tabela que nós definirmos. Cada classe
herda de uma classe base abstrata, que é sobrescrita toda vez que pedimos ao Symfony que gere
as classes de acordo com a especificação. Nós temos que trabalhar em cima dessas classes que
herdam as classes bases abstratas, pois essas últimas é que o Symfony sempre vai sobrescrever.

Pra ficar mais claro, damos uma olhada no diagrama da Figura ~\ref{fig:diag_classes_base}. Ela
apresenta todas as classes base que o Symfony gerou automaticamente para a aplicação deste trabalho.
Para cada classe BaseEntidade, há uma classe Entidade, como exposto na Figura ~\ref{fig:diag_classes_principais}, 
que é criada uma única vez, caso ela ainda não exista. É nesta última que colocamos os métodos da nossa 
lógica de negócios. Tendo isso em mente, podemos prosseguir na explicação dos diagramas.

\subsection{Diagrama de classes}

A Figura ~\ref{fig:diag_classes_base} apresenta o diagrama de classes do Mono-Manager. Todas as classes
herdam direta ou indiretamente da classe sfDoctrineRecord, pertencente ao framework ORM do Symfony. 
Essa associação não foi explicitada no diagrama para que ele ficasse mais claro. A classe sfDoctrineRecord
é uma classe abstrata que possui todos os métodos de persistência dos objetos.

\begin{figure}[htbp]
\centering
\includegraphics[width=1\textwidth]{fig/uml_classes_base.png}
\caption{Diagrama de classes base}
\label{fig:diag_classes_base}
\end{figure}

\begin{figure}[htbp]
\centering
\includegraphics[width=0.7\textwidth]{fig/uml_classes_principais.png}
\caption{Diagrama de classes extendidas}
\label{fig:diag_classes_principais}
\end{figure}

A Figura ~\ref{fig:diag_classes_principais} apresenta as classes das entidades de trabalho, 
isto é, aquelas que herdam das classes base. Nelas é que deve ser inserida qualquer lógica associada
à entidade, pois o framework não escreve nessas classes, a não ser para criá-las pela primeira vez.
As classes Proposta e Defesa possuem o método audit, que é invocado quando da avaliação, por parte da comissão,
da proposta ou da solicitação de defesa. Cada vez que um professor integrante da comissão dá seu parecer,
este método salva as informações do integrante, seu parecer e um possível comentário feito por ele
sobre a proposta/defesa. Além disso, ele calcula o parecer final dependendo da decisão da maioria 
da comissão. Essas duas entidades também guardam uma enumeração que indica seu status, que podem ser os
seguintes:

\begin{itemize}
\item NAO\_ANALISADO: Proposta/Defesa não analisada
\item APROVADO: Proposta/Defesa aprovada pelo orientador
\item REPROVADO: Proposta/Defesa reprovada pelo orientador
\item LIBERADO: Proposta/Defesa aprovada pela comissão
\item NAO\_LIBERADO: Proposta/Defesa reprovada pela comissão
\end{itemize}

A classe Projeto possui alguns métodos de verificação de situação. Os nomes dos métodos são bem 
autodescritivos e indicam se o projeto possui uma proposta/defesa, se o estudante anexou o documento
da proposta ou a cópia da monografia na solicitação de defesa, e se o projeto já pode ser defendido.

\begin{figure}[htbp]
\centering
\includegraphics[width=0.7\textwidth]{fig/uml_controllers.png}
\caption{Diagrama de classes da camada de controle}
\label{fig:diag_controllers}
\end{figure}

A Figura ~\ref{fig:diag_controllers} apresenta o diagrama de classes para a camada de controle.
A classe sfActions pertence ao Symfony e cuida da comunicação entre o navegador e a aplicação. 
Internamente é ela que instancia as classes necessárias para a execução do framework; decodifica
a \sigla{URL}{Uniform Resource Locator} da requisição, de forma a determinar qual ação e qual módulo estão 
sendo requisitados, e se não existirem, exibe uma mensagem de página não encontrada; recebe os
parâmetros de entrada; chama a ação do módulo requisitado e renderiza a saída. 

Na classe de controle monomActions encontram-se métodos de \sigla{CRUD}{Create, Retrieve, Update e Delete} 
padronizados, e que são usados pela maior parte dos módulos da aplicação. A classe de controle
documentoActions contém métodos de controle de upload de arquivos, que são usados nos módulos de 
proposta e defesa. Dessa forma, foi possível fazer um grande reuso de código.


\subsubsection{Autenticação de usuários}
Na Figura ~\ref{fig:diag_classes_base} podemos observar a classe sfGuardUser, da qual a classe Usuario extende.
Ela vem de um plugin para o Symfony de controle de usuários, que oferece recursos de segurança
e autorização em cima dos recursos de segurança oferecidos pelo framework. Aproveitamos esse plugin
para fazer um bom reuso de código e contar com toda a segurança que o plugin já disponibiliza \cite{sfguardplugin}.

As classes Estudante e Professor possuem um relacionamento de composição com a classe Usuario.
Inicialmente foi pensado em fazer com que essas classes herdassem da classe Usuario, mas visto
que existe o usuário administrador que não é nem estudante, nem professor, a modelagem não pôde
ser feita dessa forma. De qualquer forma, só podem existir estudantes e professores se houver um usuário
associado a eles.

Na classe Usuario encontram-se os dados mais básicos de cada usuário, como nome, endereço de email e senha.
Há ainda outras informações que dizem respeito a como a senha é criptografada no banco de dados. 
O campo algorithm contém o algoritmo utilizado para criptografar a senha, que por padrão é o SHA1. 
Outros algoritmos podem ser utilizados, como o MD5, mas optou-se por deixar o algoritmo padrão.
O campo salt é uma string que é concatenada à senha antes da criptografia, para dificultar a a 
descoberta do valor original.

\subsubsection{Formulários}
Além de gerar as entidades, o Symfony também possui a funcionalidade de gerar formulários que
possuem validação própria. Dessa forma, só é necessário ajustar detalhes de aparência ou
validações mais complexas, como datas que devem preceder umas às outras ou valores que devem 
estar dentro de um conjunto fechado.

\begin{figure}[htbp]
\centering
\includegraphics[width=0.5\textwidth]{fig/uml_forms.png}
\caption{Diagrama de classes dos formulários}
\label{fig:diag_forms}
\end{figure}

A Figura ~\ref{fig:diag_forms} apresenta os formulários que foram gerados a partir da descrição 
do esquema do banco de dados. Todas as classes desse diagrama herdam das suas correspondentes 
classes base, assim como acontece com as classes da Figura ~\ref{fig:diag_classes_principais}.
Elas não são apresentadas aqui por efeito de clareza no diagrama. As classes base, por sua vez,
herdam da classe BaseFormDoctrine, do framework, que possui os métodos básicos de validação e 
persistência dos formulários. 

O formulário do ANEXO ~\ref{anx:proposta} é representado no sistema pelos formulários de estudante, 
professor, projeto e proposta. O formulário do ANEXO ~\ref{anx:defesa} é representado no sistema 
pelos formulários de estudante, professor, projeto e defesa. 




\chapter{Utilização do sistema}
\label{cha:utilizacao}

\chapter{Conclusões e trabalhos futuros}
\label{cha:conclusoes}


\bibliographystyle{abnt-alf}
\bibliography{bib}

\chapter{Anexos}
\clearpage
\section{Formulário de proposta de projeto final}
\label{anx:proposta}
\begin{figure}[htbp]
\centering
\includegraphics[scale=0.6]{requisitos/Formulario_Proposta_Projeto_Final.pdf}
\end{figure}

\clearpage
\section{Formulário de solicitação de defesa e banca}
\label{anx:defesa}
\begin{figure}[htbp]
\centering
\includegraphics[scale=0.6]{requisitos/Formulario_Solicitacao_Defesa_e_Banca_v1.pdf}
\end{figure}



\end{document}

