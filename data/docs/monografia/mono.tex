% Modelo de monografia em LaTeX da UECE
%
% Na criação deste modelo foi tomada como base a dissertação de mestrado do
% Jeandro Bezerra, sem a ajuda dele este trabalho teria sido muito mais
% difícil. Este modelo utiliza o abnTeX e um pacote (uece.sty) para formatação
% de alguns anexos necessários da UECE (folha de rosto, CIP, epígrafe, ...).
%
% Este documento não clama possuir conformidade de 100\% com as normas de
% trabalhos da UECE. Consulte os guias oficiais.
%
% Autor do modelo: Rudy Matela
% Data do modelo: 20090920
% 
% Autor: Wesley Alcoforado
% Data: 22/02/2011

\documentclass[pnumabnt,normaltoc,espacoumemeio,capchap]{abnt}		
\usepackage[brazil]{babel}
\usepackage[utf8]{inputenc}
\usepackage{abnt-alf}
\usepackage{graphicx}
\usepackage{uece}
\usepackage{multicol}
\usepackage{lastpage}
\usepackage{enumerate}
\setcounter{secnumdepth}{3}
\setcounter{tocdepth}{3}
\usepackage[final]{pdfpages}

% Informações gerais do documento
\autor{Wesley Jefferson Oliveira Alcoforado}
\autorr{Alcoforado, Wesley Jefferson Oliveira}
\titulo{Mono-Manager - um gerenciador de trabalhos de conclusão de curso para o curso de Ciências de Computação da UECE}
\local{Fortaleza, Ceará}
\cidade{Fortaleza}
\data{2011}
\orientador{Mariela Cortés}
%\coorientador{Beltrano das Tantas e \par Cicrano da Silva}
\codigocip{A000z}{CDD:000.0}

% Descrição para folha de rosto
\comentario{
Monografia submetida à coordenação do curso de Ciências da Computação da Universidade Estadual do
Ceará, no ano de 2011, como requisito parcial para obtenção do grau de Bacharel em Ciências da Computação.
}

% Informações institucionais
\centro{Centro de Ciências e Tecnologia}
\curso{Graduação em Ciências da Computação}
\instituicao{Universidade Estadual do Ceará}

% Epígrafe: citação e autor
\epigrafe{``L'ordinateur obéit à vos ordres, pas à vos intentions.''}
\autorepigrafe{Anônimo}

% Membros da comissão avaliadora
\bancaum{Profª. Drª. \ABNTorientadordata\\Universidade Estadual do Ceará - UECE\\Orientador}
\bancadois{Prof. Dr. Beltrano das Tantas\\Universidade Estadual do Ceará - UECE\\Co-orientador}
\bancatres{Prof. Me. Cricrano da Silva\\Universidade Estadual do Ceará - UECE\\Co-orientador}
\bancaquatro{Prof. Dr. Zé Ninguém\\Universidade Estadual do Ceará - UECE}

% Palavras chave
\pcs{TCC}{Processo}{Informatização}
\kws{TCC}{Processo}{Informatização}

\begin{document}

\capa
\folhaderosto
\makecippage
\termodeaprovacao

\pretextualchapter{Agradecimentos}
À minha esposa Raquel por ter me incentivado a escrever minha monografia, me apoiando nos
momentos em que fraquejei.

Aos amigos de graduação mais próximos pelo apoio mútuo durante toda a jornada na universidade.

Aos professores por terem compartilhado seus conhecimentos conosco.
\pagebreak

\makeepigrafe
\tableofcontents
\listadefiguras
\listadetabelas
% \listadesiglas % \sigla{sigla}{Descrição}
% \listadesimbolos % \simbolo{símbolo}{Descrição}

\begin{resumo}
\noindent
Este trabalho tem como objetivo apresentar uma ferramenta que informatiza
o processo de avaliação de trabalhos de conclusão da graduação em Ciências 
da Computação da Universidade Estadual do Ceará. Essa ferramenta foi projetada 
para minimizar a quantidade de documentos impressos e para proporcionar uma 
visão geral do andamento das atividades. Além disso, ela é capaz de notificar 
as pessoas envolvidas automaticamente, alertando sobre o cumprimento dos prazos.
\noindent
\palavraschave
\end{resumo}
\pagebreak

\chapter{Introdução}
Através da informatização de processos, é possível otimizar tarefas que demandam tempo, 
um bem escasso, e também diminuir a quantidade de documentos impressos nas nossas mesas, 
papéis que muitas vezes se perdem no meio de outros documentos.

Segundo \cite{Freitas:1997}, sistemas de informação são mecanismos cuja função é coletar, guardar e 
distribuir informações para suportar as funções gerenciais e operacionais das organizações.
No contexto da universidade, temos muitos procedimentos padronizados por regulamentos estabelecidos
pela instituição que poderiam ser controlados por sistemas de informação, diminuindo a
dependência humana, que naturalmente está sujeita a erros e atrasos, economizando recursos humanos, 
assim como recursos materiais e centralizando a informação. 

Com a manutenção dos dados em uma mesma base é possível ter uma visão geral do presente 
e do passado e analisar meios de melhorar ainda mais o processo. Por esse motivo, sistemas de apoio
computacional são extremamente importantes, pois ajudam a coletar dados que de outra forma estariam
dispersos, e que devido ao pouco tempo disponível dos coordenadores, a relação custo-benefício dessa
coleta e análise manual seria desestimulante.

\section{Motivação}
A disciplina de Projeto Final que faz parte do último semestre do curso de Ciências da Computação
tem por objetivo o desenvolvimento do TCC, que é um pré-requisito para a obtenção do título de Bacharel.

Considerando a importância da disciplina neste cenário, o desenvolvimento dos trabalhos é regido por um
regulamento que define o processo de acompanhamento necessário para atingir os objetivos propostos.
Todo aluno matriculado nessa disciplina deve seguir a regulamentação específica. Idealmente todo o 
processo pode ser informatizado, de forma a organizar melhor as tarefas dos professores e alunos envolvidos.

A construção de um sistema que gerencie o processo de submissão de projetos ajudaria 
não apenas na organização dos documentos e datas, mas também possibilitaria aos 
orientadores e à Comissão de Projeto Final ter um controle dos alunos que, por um 
motivo ou outro, não sabem por onde começar ou estão parados no meio do caminho, 
além de manter todos informados sobre o andamento dos seus projetos.

\section{Objetivo}
Com o desenvolvimento de uma ferramenta de apoio que gerencie a submissão de TCCs, espera-se
uma diminuição no esforço despendido no controle do processo, acesso centralizado a documentos e
datas, tanto para os estudantes e seus orientadores, quanto para a Comissão de Projeto Final.

Também espera-se fornecer à Comissão uma visão geral do andamento de todos os TCCs, quais foram
iniciados, quais estão pendentes, quais serão concluídos em breve, quais estudantes já
defenderam mas ainda não entregaram a versão final das suas monografias e um histórico/prospecção
de quantos estudantes se formaram/formarão por semestre.

\subsection{Objetivos específicos}
Para conseguir desenvolver esta ferramenta de apoio, é necessário levantar os requisitos
junto aos principais envolvidos, escrever casos de uso que sejam detalhados apenas o suficiente
para descrever as funcionalidades do sistema, evitando perda de tempo com documentação inútil. 
Após o desenvolvimento são necessários testes na ferramenta e elaboração de um manual de 
apoio com sua instalação e utilização.

\section{Metodologia}

Para o desenvolvimento deste trabalho foi seguido o modelo de desenvolvimento em cascata que é
dividido nas seguintes etapas, segundo \cite{sommerville:2003}:
\begin{itemize}
\item Análise e definição de requisitos 
\item Projeto de sistemas e de software 
\item Implementação e testes de unidade 
\item Integração e testes de sistema 
\item Operação e Manutenção 
\end{itemize}
 
Na fase de análise e definição de requisitos foram construídos casos de usos 
simplificados, onde as funcionalidades foram expostas com mais clareza, mas sem 
muito aprofundamento para garantir que o projeto fosse concluído em tempo hábil.
A partir dos casos de uso, o banco de dados foi modelado, assim como as classes 
de acesso a dados, o que corresponde à etapa de Projeto. Após o término da modelagem,
foi iniciada a fase de codificação dos casos de uso, seguidos pelo desenvolvimento 
do design da aplicação. Neste passo não foram feitos testes de unidade codificados,
apenas testes manuais. Após o desenvolvimento foi feita a integração dos módulos 
e os testes de sistema.

No momento da elaboração desta monografia o sistema ainda não se encontra em uso,
mas espera-se que o próximo passo seja o último passo do modelo cascata, a operação
e manutenção.

O modelo cascata, apesar de inflexível, foi bastante apropriado pois havia um entendimento
claro dos requisitos do sistema e ofereceu uma previsibilidade maior dos prazos.

Para a construção da aplicação foi utilizada a linguagem de script \sigla{PHP}{PHP: Hypertext Preprocessor}, na 
sua versão 5.3.3 em conjunto com o framework Symfony 1.4.2. Como IDE, foi utilizado 
o NetBeans 6.9, um framework para desenvolvimento e testes com PHP. O 
banco de dados escolhido foi o PostgreSQL versão 8.4.

Estas ferramentas foram escolhidas devido a serem gratuitas e possuirem versões 
estáveis para Linux, que é a plataforma onde se pretende implantar a versão 
final da aplicação. Mais detalhes sobre as tecnologias são expostas na Seção ~\ref{tecnologias}.

\section{Organização do trabalho}

Além deste capítulo introdutório, o presente trabalho consiste em mais 4 capítulos. 
No Capítulo ~\ref{cha:regulamento} é feita uma apresentação do processo de 
entrega de \sigla{TCC}{Trabalho de Conclusão de Curso}s na UECE, de acordo
com o regulamento da instituição. No Capítulo ~\ref{cha:desenvolvimento} é descrito como foi o processo de desenvolvimento 
da aplicação, assim como a organização do banco de dados e os requisitos do sistema. 
O Capítulo ~\ref{cha:utilizacao} demonstra como utilizar o sistema. O Capítulo ~\ref{cha:conclusoes} apresenta as conclusões
e uma breve discussão sobre trabalhos futuros.


\chapter{Regulamento de Projeto Final}
\label{cha:regulamento}

O processo de submissão e avaliação de projetos finais no curso
de Ciências da Computa\-ção da UECE segue um regulamento que normatiza 
o tipo do conteúdo da monografia, o procedimento para o desenvolvimento 
e para a aprovação do projeto, a constituição da Comissão de Projeto 
Final e as atribuições desta, do orientador e do aluno. 

\section{Entidades envolvidas e suas atribuições}
O desenvolvimento do projeto final deve ser desempenhado individualmente
pelo aluno, sob a orientação de um docente, o orientador. O orientador deve
ser um docente lotado no curso de Ciências da Computação da UECE, seja ele
professor efetivo, substituto ou visitante. O aluno pode ainda contar com a
colaboração de co-orientadores, podendo estes serem docentes da UECE ou de 
outras \sigla{IES}{Instituição de Ensino Superior} ou ainda profissionais com graduação
plena em Ciências da Computação ou cursos afins e com no mínimo 3 (três) anos
de experiência em orientação de alunos ou coordenação de projetos.

A Comissão de Projeto Final, ou simplesmente Comissão, é o órgão 
responsável pelo acompanhamento do processo de desenvolvimento do projeto final. 
Ela é composta por 3 (três) docentes efetivos pertencentes ao curso de 
Bacharelado em Ciência da Computação da UECE, havendo ainda 2 (dois) membros 
suplentes, sendo todos esses (permanentes e suplentes) escolhidos através de 
eleição no Colegiado do curso de Ciência da Computação e nomeados pelo 
Coordenador da graduação. O membro da Comissão fica impedido de emitir 
parecer sobre o trabalho de seus orientandos, que, neste caso, 
deverão ser avaliados por um membro suplente.

Compete ao aluno:
\begin{enumerate}[a.]
\item elaborar projeto de Proposta de Projeto Final;
\item conduzir e executar o Projeto Final;
\item cumprir os prazos estabelecidos no cronograma pré-estabelecido;
\item redigir e defender o Projeto Final;
\item entregar cópia corrigida do Projeto Final à secretaria;
\item tomar ciência dos prazos estabelecidos pela Comissão de Projeto Final e cumpri-los.
\end{enumerate}

Compete ao orientador e co-orientador:
\begin{enumerate}[a.]
\item orientar o aluno na organização de seu plano de estudo, pesquisa e assistí-lo na preparação da monografia;
\item viabilizar a realização do Projeto Final;
\item encaminhar a Proposta de Projeto Final e a Solicitação de Defesa à Comissão;
\item propor à Comissão a composição da Banca Examinadora;
\item encaminhar a Ata de Defesa, devidamente preenchida e assinada, ao Coordenador do curso.
\end{enumerate}

Compete à Comissão:
\begin{enumerate}[a.]
\item aprovar a proposta e plano de trabalho de Projeto Final;
\item aprovar as indicações dos orientadores de Projeto Final que não sejam docentes do curso;
\item aprovar os membros das bancas avaliadoras do Projeto Final;
\item autorizar a defesa de monografia de Projeto Final;
\end{enumerate}

\section{Desenvolvimento do projeto}

O projeto pode ser iniciado antes do aluno se matricular na disciplina de Projeto Final, porém o 
processo de desenvolvimento do trabalho deve ser realizado no último ano do curso. O desenvolvimento
do projeto final é constituido das seguintes partes:

\begin{enumerate}[a.]
\item Apresentação da proposta de projeto final à comissão de projeto final;
\item Solicitação de defesa do projeto final e indicação de comissão examinadora à comissão de projeto final;
\item Defesa do projeto final em seção pública diante da comissão examinadora;
\item Entrega do texto final da monografia.
\end{enumerate}


\subsection{Proposta do projeto} 
O aluno deve apresentar uma proposta de projeto (ver Anexo ~\ref{anx:proposta}) a partir do início do período
letivo em que se matriculou na disciplina de Projeto Final. A data máxima de apresentação da
proposta é de 100 (cem) dias antes da data da colação oficial ou especial.

A apresentação da proposta deve ter a anuência do orientador e é avaliada pela comissão. Após a 
aprovação da proposta pela comissão, fica autorizado o início do desenvolvimento do trabalho.
A proposta deve conter os seguintes tópicos:

\begin{enumerate}[a.]
\item Motivação e Objetivo;
\item Fundamentação teórica;
\item Metodologia;
\item Bibliografia;
\item Cronograma.
\end{enumerate}

\subsection{Defesa do projeto}
O aluno, com a anuência do orientador, deve encaminhar uma solicitação de defesa
à comissão (ver Anexo ~\ref{anx:defesa}), no mínimo 60 (sessenta) dias após a aprovação da proposta e no máximo 
até 30 (trinta) dias antes da colação oficial ou especial.
Junto da solicitação deverão seguir a data de sugestão da defesa, a indicação
dos membros da comissão examinadora e 1 (uma) cópia da monografia.

Sendo o parecer da comissão favorável, o aluno tem um prazo de até 10 (dez)
dias antes da colação oficial ou especial para realizar a defesa, e 7 (sete) dias
antes desta para entregar à comissão examinadora exemplares da monografia.

Sendo o aluno aprovado na defesa, ele deverá entregar à secretaria do curso 3 (três)
exemplares impressos da monografia e 1 (uma) cópia em meio eletrônico. Somente após 
isso é que será autorizada a emissão e entrega do diploma ao aluno.

\subsection{Resumo do fluxo do desenvolvimento do projeto final}
A Figura ~\ref{fig:flux_tcc} apresenta um fluxo relativo ao processo de desenvolvimento do projeto final. 
Inicialmente, o aluno submete uma proposta de acordo com o modelo contido no Anexo ~\ref{anx:proposta}, 
que deve ter a anuência do seu orientador. Caso o orientador não aprove a proposta, convém que
o aluno converse com seu orientador para corrigir os eventuais problemas para que dessa forma possa
submeter a proposta novamente. Após obter o aval do orientador, a proposta pode ser encaminhada
à comissão. Surgindo algum problema na aprovação da proposta por parte da comissão, o aluno
deve corrigir os problemas indicados e submetê-la novamente. Caso a proposta seja aprovada, o aluno
pode começar a desenvolver o trabalho. Finalizado o desenvolvimento, o aluno pode solicitar 
a defesa do seu projeto enviando 1 (uma) cópia da monografia e preenchendo o formulário do
Anexo ~\ref{anx:defesa}. O pedido deve novamente contar com a anuência do orientador, para depois
seguir para a comissão. Sendo a defesa aprovada pela comissão, o aluno está apto para defender
seu projeto ante a comissão examinadora.


\begin{figure}[htbp]
\centering
\includegraphics[width=1\textwidth]{fig/fluxograma_tcc.png}
\caption{Fluxograma do desenvolvimento do projeto final}
\label{fig:flux_tcc}
\end{figure}



\bibliographystyle{abnt-alf}
\bibliography{bib}
\appendix
\chapter{Anexos}
\clearpage
\section{Formulário de proposta de projeto final}
\label{anx:proposta}
\begin{figure}[htbp]
\centering
\includegraphics[scale=0.6]{requisitos/Formulario_Proposta_Projeto_Final.pdf}
\end{figure}

\clearpage
\section{Formulário de solicitação de defesa e banca}
\label{anx:defesa}
\begin{figure}[htbp]
\centering
\includegraphics[scale=0.6]{requisitos/Formulario_Solicitacao_Defesa_e_Banca_v1.pdf}
\end{figure}



\chapter{Formulários}
\section{Formulário de proposta de projeto final}
\includepdf[pages=1,noautoscale=true,scale=0.75,frame=true]{requisitos/Formulario_Proposta_Projeto_Final.pdf}

\end{document}

