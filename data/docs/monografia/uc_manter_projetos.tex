\begin{longtable}{r p{12cm}}
\hline
Atores & Estudante \\ \hline
Pré-condições & O estudante deve estar logado no sistema.\\
Fluxo básico &  1. O caso de uso se inicia quando o estudante seleciona manter projetos no menu do sistema. \newline
                2. Uma vez que o estudante seleciona uma das opções disponíveis (incluir, alterar, excluir, listar, visualizar comentários): \newline
                \hspace*{1cm} a) Se o estudante selecionar a opção incluir, o caso de uso segue para o sub-fluxo 2 - Incluir Projeto. \newline 
                \hspace*{1cm} b) Se o estudante selecionar a opção alterar, o caso de uso segue para o sub-fluxo 3 - Alterar Projeto.  \newline 
                \hspace*{1cm} c) Se o estudante selecionar a opção excluir, o caso de uso segue para o sub-fluxo 4 - Excluir Projeto.  \newline 
                \hspace*{1cm} d) Se o estudante selecionar a opção listar, o caso de uso segue para o sub-fluxo 5 - Listar Projetos.  \newline 
                \hspace*{1cm} e) Se o estudante selecionar a opção visualizar comentários, o caso de uso segue para o sub-fluxo 6 - Visualizar Comentarios.  \newline 
                3. O caso de uso se encerra. \newline \\
Incluir Projeto & 1. Este sub-fluxo se inicia quando o estudante seleciona incluir um novo projeto. \newline
                    2. O sistema exibe os seguintes campos (os campos com asterisco são obrigatórios): \newline
                    \hspace*{1cm} * Titulo \newline
                    \hspace*{1cm} * Orientador \newline
                    \hspace*{1cm} * Coorientadores \newline
                    3. O estudante preenche os campos e seleciona a opção salvar. \newline
                    4. O sistema valida se os campos obrigatórios foram preenchidos. \newline
                    5. O sistema inclui o projeto no banco de dados. \newline
                    6. O caso de uso se encerra. \newline \\
Alterar Projeto & Pré-condições: O estudante deve ter selecionado um projeto para a alteração. \newline
                    1. Este sub-fluxo se inicia quando o estudante seleciona alterar projeto.  \newline       
                    2. O sistema exibe os campos preenchidos.  \newline
                    3. O estudante altera os dados e solicita salvar os dados. \newline
                    4. O sistema valida se os campos obrigatórios foram preenchidos. \newline
                    5. O sistema salva as alterações no banco de dados. \newline
                    6. O caso de uso se encerra. \newline \\
Excluir Projeto & Pré-condições: O estudante deve ter selecionado um projeto para a exclusão. \newline
                    1. Este sub-fluxo se inicia quando o estudante seleciona excluir projeto. \newline
                    2. O sistema solicita que o projeto confirme a exclusão. \newline
                    3. O estudante confirma a mensagem. \newline
                    4. O sistema exclui o projeto do banco de dados. \newline
                    5. O caso de uso se encerra. \newline \\
Listar Projetos & 1. Este sub-fluxo se inicia quando o estudante seleciona listar projetos. \newline
                     2. O sistema exibe a listagem dos projetos, contendo os seguintes campos:\newline
                     \hspace*{1cm} a) Orientador\newline
                     \hspace*{1cm} b) Título\newline
                     \hspace*{1cm} c) Proposta\newline
                     \hspace*{1cm} d) Defesa\newline
                     \hspace*{1cm} e) Status - Status mais atual da defesa, ou se esta não tiver sido iniciada, o status mais atual da proposta.\newline
                     3. O caso de uso se encerra.\newline     \\ 
Visualizar Comentários & 1. Este sub-fluxo se inicia quando o estudante solicita visualizar os comentários. \newline
                     2. Uma vez que o estudante seleciona uma das opções disponíveis (comentários do orientador, da comissão):\newline
                     \hspace*{1cm} a) Se o estudante selecionar visualizar os comentários do orientador, o sistema exibe uma listagem com os comentários do orientador.\newline
                     \hspace*{1cm} b) Se o estudante selecionar visualizar os comentários da comissão, o sistema exibe uma listagem com os comentários da comissão.\newline
                     3. O caso de uso se encerra.\newline                                        
               \\ \hline
Fluxos alternativos & Dados obrigatórios não preenchidos  \newline
                        1. Este sub-fluxo se inicia no passo 4 dos sub-fluxos Incluir Estudante e Alterar Estudante, quando o usuario não informou todos os campos obrigatórios. \newline
                        2. O sistema exibe ao lado do campo uma mensagem de que o campo deve ser preenchido e aguarda até que o administrador o preencha. \newline
                        3. O subfluxo segue para o passo 2 do sub-fluxo do qual ele se originou. \newline
                    \\ \hline        
\end{longtable}





